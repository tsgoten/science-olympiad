\documentclass{article}
\author{SBHS Science Olympiad, Tarang}
\usepackage[margin=.1in, landscape]{geometry}
\setlength{\columnsep}{.2in}
\usepackage{amsmath, multicol}
\usepackage{xcolor}

% Colors
\newcommand{\red}[1]{\textcolor{red}{#1}}
\newcommand{\green}[1]{\textcolor{green}{#1}}
\newcommand{\blue}[1]{\textcolor{blue}{#1}}
\newcommand{\pink}[1]{\textcolor{magenta}{#1}}
\newcommand{\orange}[1]{\textcolor{orange}{#1}}
\newcommand{\yellow}[1]{\textcolor{yellow}{#1}}
% Headings
% Note: The order of importance is the ROYGBV
\newcommand{\mysection}[1]{\textbf{\textit{\red{#1}}}}
\newcommand{\mysubsection}[1]{{\textit{\orange{#1}}}}
\newcommand{\mysubsub}[1]{{{\green{#1}}}}
\newcommand{\mysubsubsub}[1]{{{\blue{#1}}}}
\newcommand{\vocab}[1]{{\pink{#1}}}
% Pictures

\begin{document}
	% Uncomment the line below to modify the font size
	% \tiny
	\begin{multicols*}{4}
		
		\noindent
		
		\mysection{General Ecology} \\
		Ecology: how organisms interact with one another and with their environment \\
		Environment: abiotic and biotic features\\
		\textit{Levels of Organization:} Population (same species) $>$  Community (diff. species, biotic) $>$ Ecosystem (community+abiotic) $>$ Biosphere (portion of Earth w/living species)
		\\
		\mysection{Aquatic Ecosystems} \\
		\mysubsection{Lentic Ecosystems} (STILL Water) \\
		\mysubsub{Ponds} Bottom of the pond still receives light, unlike lakes.
		\mysubsub{Horizontal Lake Zones}
		\mysubsubsub{Littoral Zone}: Near the shoreline; Sunlight penetrates all the way to sediments; Allows for aquatic plants (\vocab{macrophytes}) to grow.
		\mysubsubsub{Limentic Zone} open water, away from shore.
		\mysubsub{Vertical Lake Zones}
		
		
	\end{multicols*}
\end{document}